
\section{Introduction}

The field of computational neuroscience has provided
many \e{systems models} of the brain \citep{arai_two-dimensional_1994,gancarz_neural_1998,hazy_towards_2007,blenkinsop_frequency_2017}.
We refer to these as \e{mechanistic computational models}, meaning
models which consist of populations of neural elements, interconnected
in a biologically plausible manner, which simulate the operation of
the brain. Whilst they differ in scale and complexity, these models
all seek to
describe the fundamental mechanisms behind common animal behaviours
such as locomotion, threat evasion, reaching or feeding. However,
none of the models cited here actually reproduce these
behaviours. In each case, the activity in a certain population of
neurons is taken to be representative of a behavioural outcome.
In some cases, it \e{is} reasonable to take the activity of an internal
population within the brain model as being representative of the
induced behaviour. For example, a choice made in a \e{go/no-go} task
could be determined from activity in a population within a basal
ganglia model
\citep{nambu_discharge_1990,kuhn_event-related_2004}. The decision
to \e{go} is selected by a reduction of activity in this population;
maintenance of activity implies \e{no-go}. To validate the model, the
error rates which it generates
could be compared with experimentally determined error rates in
primate subjects. We refer
to this as an \e{output assumption model} because the output is assumed
to signify behaviour. (An \e{input assumption model} assumes that sensory
input produces some particular form of neural activity in an input
population of the model.)

However,
we may be interested in reproducing accurate simulated \e{trajectories},
in order to find out how degradation of parts of the model affect
movement. In Parkinson's Disease, degradation of the
dopamine neurons originating in the substantia nigra pars compacta (SNc)
causes diskinesia \citep{galvan_pathophysiology_2008}, as well as
abnormal network activity in
the basal ganglia \citep{brown_dopamine_2001,mccarthy_striatal_2011}.
Sufferers of the disease would be
expected to produce abnormal decision-making \e{and} movement trajectories
in a reach-to-the-correct-target task such as the one described in
\cite{james_target-distractor_2017}.
A model which
sought to explore in detail the effects of the SNc degradation both on
the decision making \e{and} on the movement dynamics would need a
physically accurate virtual arm, as well as physically realistic sensory
input for the brain. This is no less than a complete model of those sections
of the brain and body which act to fulfil the task. Such a modelling
effort, if successful, would result in a virtual robot capable of expressing
behaviour \e{in response to sensory input from its environment}. This would
represent a paradigm shift in the field of computational neuroscience
worthy of the new name of \e{computational neurobehaviour}.

In an attempt to build a model combining brain, realistic
biomechanics \e{and} sensory
feedback, we sought to extend our previous work modelling the oculomotor system.
The existing model \citep{cope_basal_2017} is already able to capture sensory input
and convert it into a neural signal, assumed to specify the target of a
\e{saccadic eye movement};
a fast movement of the eyes which directs the fovea to a region of interest
in the field of view.
The oculomotor system is an excellent candidate for modelling because
its movements can  be specified with only three degrees of freedom,
making it one of the simplest neuro-muscular system in the body.
It is nevertheless behaviourally
interesting, as saccadic eye movements reveal information about decision making at
a subconscious level
\citep{deubel_saccade_1996,reppert_modulation_2015,marcos_determining_2016}.
The modelling of the oculomotor system is served by a large body of
behavioural data describing
saccades \citep{tipper_reaching_2001,walker_effect_1997,casteau_effect_2012},
many anatomical studies of the neural substrates involved
\citep{meredith_intrinsic_1998,isa_intrinsic_2002,isa_exploring_2009}
and electrophysiological data linking these together
\citep{hepp_spatio-temporal_1983,dorris_neuronal_1997,mcpeek_competition_2003,vokoun_circuit_2011}.
Furthermore, in the context
of building \e{behaving} systems a
necessary part of any model for which the behaviour requires visual attention
and decision making is a realistic mechanism for gathering visual information.
This is obvious from extrinsic considerations---a
subject must look at a scene to make decisions or navigate within it. It
also follows for \e{intrinsic} reasons. For example, \cite{howard_hand_1997}
showed that visual cues affect reach trajectories and the same group later
demonstrated that reaching affects the saccadic system
\citep{tipper_reaching_2001} suggesting a close relationship between these
neural systems. Building the simplest, behaving oculomotor system will
therefore assist future computational neurobehavioural modelling efforts.

Many neural populations are involved in the coding of saccadic eye
movements, only a very brief overview is given here; for a
review, see \cite{munoz_commentary:_2002}. One pathway takes
information from the retina directly into the superficial layers of
the superior colliculus in the brainstem
\citep{sterling_receptive_1971,linden_massive_1983,wu_involvement_1994}.
Activity within
the superior colliculus then excites neurons in the pons, medulla and
rostral mid-brain \citep{sparks_brainstem_2002} and then through
channels to the motor neurons which innervate the extraocular muscles
\citep{fuchs_firing_1970,sparks_brainstem_2002}. This direct pathway
is responsible for the low latency saccades called express saccades
\citep{schiller_effect_1987,edelman_activity_1996}.
Information from the retina is also processed by visual cortex which
feeds through to the frontal eye fields in which activity is related
to reflexive and voluntary
saccades \citep{schall_neural_1999}. Activity build-up in the frontal
eye fields is transferred to the intermediate layers of the superior
colliculus
\citep{stanton_frontal_1988-1} and is also processed by the basal ganglia,
which participates in the selection of the winning saccade end point
\citep{stanton_frontal_1988}.
Although both cortical and subcortical paths produce a saccade
target signal in the superior colliculus, it is also possible for
animals to make relatively normal saccades even after the
colliculus has been ablated
\citep{wurtz_activity_1972,aizawa_reversible_1998},
though express saccades are lost with collicular lesions
\citep{schiller_effect_1987}. This makes the superior colliculus
a perplexing structure, being both critically involved in saccade
target specification \citep{sparks_sensory_1987} and saccade dynamic control
\citep{waitzman_superior_1991,goossens_optimal_2012} and yet dispensible.
The `backup pathway' likely incorporates the oculomotor vermis and
fastigial oculomotor region of the
cerebellum which are known to participate in the specification, dynamics
and adaptation of saccadic eye movements
\citep{kleine_saccade-related_2003,takagi_effects_1998}.

There is a long history of modelling the oculomotor system. For a comprehensive
review, see \cite{girard_brainstem_2005}. Models of individual systems have been proposed for brainstem \citep{robinson_oculomotor_1975,scudder_new_1988,gancarz_neural_1998},
cerebellum \citep{quaia_model_1999,dean_modelling_1995,dean_learning_1994} and
superior colliculus
\citep{arai_two-dimensional_1994,moren_mechanism_2013,marino_spatial_2012}.
More recently, combined models have also been
developed incorporating sensory input \citep{cope_basal_2017} and driving simple
physical plants representing the eye \citep{tabareau_geometry_2007,nguyen_saccade_2014,thurat_biomimetic_2015}.
None of these models has yet fully closed the loop to produce a behaving system
operating freely within its environment. We argue that developing
integrated, closed-loop models of behaving systems offers insights into
the operation of neural systems that are not available from
input-~or output-assumption models.


\section{Material \& Methods}

The integrated brain and biomechanical model described here is a
development of the model in \cite{cope_basal_2017},
referred to here as the \ccg~model. This was a rate-coded neural
network model incorporating retinal populations, frontal eye fields
(FEF), the basal ganglia (BG), and the superior colliculus (SC). In
the \ccg~model, the centroid of the activity in the deep
layers of superior colliculus was assumed to accurately encode the
location of the eye at the end of the
saccade \citep{wurtz_activity_1972,robinson_eye_1972,van_gisbergen_collicular_1987,mcilwain_lateral_1982}.
This location was used to recalculate the positions of the luminances in
the eye's frame of reference at each time step. The model included no
brainstem populations other than superior colliculus, nor a
neuromuscular model.

To summarize, the \ccg~model takes as \e{input} the positions
of luminances on a topographic map and produces as output a saccade
target.

To the \ccg~model, we added a rate-coded brainstem model (a `saccadic
burst generator') and a biomechanical eye, implemented using the
biomechanical modelling framework OpenSim.
These will be described below, but first we will give a description of
the co-ordinate systems and the modelling framework used.

\subsubsection{Co-ordinates in the world}

Before describing the biomechanical eye and the brain model, which consisted
of retinotopically mapped neural sheets, we describe the co-ordinate system
used in the world. The eye was located at the origin of a three-dimensional, right-handed
Cartesian co-ordinate system, with its fovea directed in the $-z$ direction.
There was a notional spherical screen which was also centred at the origin of the
co-ordinate system and had a radius of 50 (in arbitrary units). The \e{fixation
point} was the point on the
screen at which the eye was initially directed.
Onto the screen were projected target luminances, each of which having a position
described by two co-ordinates; $\theta_{x}^{t}$, a
rotation of the horizon plane about the $x$ axis, and $\theta_{y}^{t}$, a rotation
of the meridian plane about the $y$ axis. The position is the intersection
of these rotated planes with the spherical screen (disregarding
the intersection point of these three surfaces behind the eye).
%
Note that a luminance with positive $\theta_{x}^{t}$ was above the horizon of this world;
one whose $\theta_{y}^{t}$ was positive lay to the left of the world's meridian. For this
reason, many of the figures in this paper are plotted with $-\theta_{y}$ on the
$x$-axis and $\theta_{x}$ on the $y$-axis so that targets that lay up and to the right
in the world do so in the graphs, also.

Luminances were crosses of height and width subtending $\pm$3\dg~and whose
`bars' were 2\dg~thick. Luminances were oriented like $+$ symbols with
their vertical bar aligned with the meridian plane and their horizontal bar
aligned with the horizon.

The eye's frame of reference was initially aligned with the world's frame
of reference.
At each timestep, the eye's rotational state (described by the Euler
rotations $\theta_x$, $\theta_y$, $\theta_z$) was
used to translate the three dimensional
Cartesian co-ordinates of the luminances in the world frame into co-ordinates
in the eye frame. The luminance co-ordinates in the eye's frame of reference
were used to determine the input to the brain model.

\subsection{Model development framework}

The \ccg~model was originally developed to run on the BRAHMS model
execution framework
\citep{mitchinson_brahms:_2010,mitchinson_brahms_2015}. To run a
BRAHMS model, the researcher must develop \e{BRAHMS components} for
the various neural elements. A BRAHMS component is a programmatically
coded implementation of the behaviour of the component. It may have an
arbitrary number of inputs and outputs and may be written in C, C++,
Python or MATLAB. The \ccg~model's components were hand written in C++
and MATLAB. A BRAHMS \e{SystemML} file describes how the different
components connect together and how data is passed between them
\citep{mitchinson_brahms:_2010}. The main BRAHMS program first
reads the SystemML file, then dynamically loads all the required
components before executing the system.

In the current work, the \ccg~model was reproduced using the
declarative SpineML markup language \citep{alex_cope_spineml_2014,richmond_model_2014},
with the help of the graphical SpineML model editing software
called SpineCreator
\citep{cope_spinecreator_2015,cope_spinecreator:_2016}. SpineML,
which is a development of the NineML specification
\citep{incf_task_force_on_multi-scale_modeling_network_2011},
describes neural populations and their projections in a highly
structured format in which neuron bodies, pre- and post-synapses are
described in terms of \e{SpineML components}. These are similar to the
components provided by BRAHMS, but in this case, the components are an
XML description of the functionality of the component, rather than a
programmatic implementation, with one XML file per component. A
SpineML \e{network layer} file then describes which components are
used in the model, and how they are connected together. Finally, a
number of SpineML \e{experiment layer} files specify how the model
described in the network layer can be executed. In the experiment
layer, the execution duration and timestep can be specified, along
with input conditions, connection lesions and component parameter
updates. A description of SpineML is given
in \cite{richmond_model_2014}; the definitive definition is found
in the schemas \citep{cope_spineml_2014}.
SpineCreator, in its r\^ole as a graphical editor for the SpineML
format, was used to generate the SpineML files describing the
model. It was also used to generate the diagrams of the model.

As a declarative format for model specification, SpineML is agnostic
about how the model is executed. A number of simulation
engines can be utilised, including DAMSON \citep{richmond_damson_2015}, GeNN
\citep{nowotny_flexible_2011,nowotny_spineml_2014} and BRAHMS (used
here). The simulation engine incorporating BRAHMS is called \stob~
\citep{cope_spineml_2_brahms_2015-1}. \stob~is a collection of XSLT
stylesheets which first generate and compile C++ BRAHMS components
from the SpineML component layer description files. \stob~then uses
the SpineML network and experiment layer files to generate a BRAHMS
SystemML description of the model. Finally, \stob~executes the model,
now described entirely as a BRAHMS system, via a call to the BRAHMS
binary. A number of additional hand-written components are present
in \stob~providing the inputs (constant inputs, time-varying inputs,
etc) which the modeller specifies in the experiment layer.

In addition to the brain model components, all of which are
code-generated using \stob~as described above, two hand-written
components are integrated into the model: The biomechanical eye model
and a sensory input component. The sensory input component takes the
eye's rotational state and the state of the experimental luminances
and projects a retinotopic activity map into the brain model. Both of
these BRAHMS components were hand-written in C++. To incorporate these
components into the SpineML model, a \stob~\e{external.xsl} file was used.
The external.xsl file scheme for incorporating external BRAHMS components
into a SpineML model was a new \stob~feature motivated by the current work.
Fig.~\ref{model_framework} shows the workflow, in which the
model specification files (blue box - a combination of SpineML files
and C++ code), are processed (green box) into a BRAHMS system (red
box).

\subsection{Existing brain model}

The brain model, excluding the brainstem, is a re-implementation of
the \ccg~model. Fig.~\ref{brain_model} shows the layout of the
populations and the interconnections between them. Each population of
2500 elements is arranged as a 50 by 50 grid which forms a map in a
retinotopic co-ordinate system, roughly matching the known layout of
the superior colliculus \citep{robinson_eye_1972}. % FIXME: and frontal eye field?

\subsubsection{Components}

With the exceptions of the World and FEF\_add\_noise populations, each
neural element represents an activation; the activation is governed by
a first order differential equation specified in the SpineML
component. In the brain model, there are six different components in
use: LINlinear; LINret; LINexp; D1MSN and D2MSN.

The LINlinear component governs the activation $a$ with a first order
leaky integrator differential equation:
\begin{equation}
   \dot{a} = \frac{1}{\tau}(a_{in}-a)
\end{equation}
where $\tau$ is the time constant for the neural activation and
$a_{in}$ is the input to the neural element. $a_{in}$ is defined by an
activation input and a shunting inhibition input according to:
\begin{equation}
   a_{in} = A(1-s_a)+\alpha R_N
\end{equation}
Here, $A$ is the activation input and $s_a$ is the shunting inhibition
state variable whose value is related to the shunting input, $S$ by
\begin{equation}
   s_a = \begin{cases}
      S & S\leq 1 \\
      1 & S > 1
   \end{cases}
\end{equation}
$R_N$ is a random number drawn from a standard normal distribution
($\sigma$=1, $\mu$=0) and introduces noise to the activation of the
neural element, with the parameter $\alpha$ controlling the noise
amplitude.

The output, $y$, of LINlinear is related to the activation $a$ by the
piecewise linear transfer function
\begin{equation}
   y(a) = \begin{cases}
      0   & a < c \\
      a-c & c \leq a \leq 1+c \\
      1   & a > 1+c
   \end{cases}
\end{equation}
where $c$ is a parameter defining the slope of the transfer
function. At this point, the naming scheme for the component becomes
apparent; this is a Leaky Integrator with a piecewise-linear transfer function.

The LINret component used for the retinal populations is similar
to the LINlinear component, but with no intrinsic noise and no
shunting inhibitory input. It has a neural input which is identical to
the activation input $A$:
\begin{equation}
   a_{in} = A
\end{equation}
The LINexp component is a leaky integrator with an exponential
transfer function. It shares the same differential equation with
LINlinear, but has a different input equation and a different output
transfer function. It has the following equation for the neural
element input $a_{in}$:
\begin{equation}
   a_{in} = [A+N(a-V_{r}^{-})] (1-S) + 0.01 R_N
\end{equation}
where $A$ is the activation input and $N$ is an input which is
modulated by $V_{r}^{-}$, a reversal potential, and $a$, the current
activation of the element. These inputs are summed and then reduced by
a factor which is dependent on $S$, the shunting input. As in
LINlinear, $R_N$ introduces normally distributed noise to the element.

The output, $y$, of the LINexp component is given by
\begin{equation}
   y(a) = \begin{cases}
      \mathrm{e}^{a}-0.9   & \mathrm{e}^a \leq 1+0.9 \\
      1   & \mathrm{e}^a > 1+0.9
   \end{cases}
\end{equation}
This component is used in the subthalamic nucleus (STN) population, as it gives a more
physiologically accurate f-I
behaviour \citep{wilson_model_2004,bevan_mechanisms_1999,hallworth_apamin-sensitive_2003}
which has been shown to allow the mapping of the basal ganglia network
architecture onto an optimal decision making
model \citep{bogacz_basal_2007}.

The D1MSN and D2MSN components are both leaky integrators, similar to
LINlinear. They differ in that they have no shunting inhibition. They
are used in to model medium spiny neuron (MSN) populations in
the striatum. As they model
the fact that most MSN neurons fall into two groups; those expressing D1
dopamine receptors and those expressing D2 receptors, they have a
dopamine parameter that modulates the input activation, so
that their equations for $a_{in}$ are thus:
\begin{equation}
   a_{in}^{D1} = (0.2 + d)A + 0.01 R_N
\end{equation}
\begin{equation}
   a_{in}^{D2} = (1 - d)A + 0.01 R_N
\end{equation}
where $d$ is the dopamine parameter. Varying dopamine from 0 to 1
enhances the activation in the D1 model, whereas it decreases
the activation of the D2 model elements, in line with experimental
observations \citep{harsing_influence_1997,gonon_prolonged_1997}.
Note that the equation for $a_{in}^{D1}$
differs from that used in the \ccg~model, for which the
cortico-striatal weights are multiplied by $(1+d)$ rather than
$(0.2+d)$.

The equations given above are applied to each element in a
population. The value of the activation $A$ (and where relevant, the
shunting input, $S$) is determined by summing the weighted
inputs to the population:
\begin{equation}
A = \sum_{i}w_i^{act} x_i^{act}
\end{equation}
\begin{equation}
S = \sum_{i}w_i^{sh} x_i^{sh}
\end{equation}
$w_i^{act}$ and $w_i^{sh}$ are, respectively, the weights
of the $i^{th}$ activation or shunting connection; $x_i^{act}$
and $x_i^{sh}$ are the signals input to the
activation and shunting connections.

\subsubsection{Population activity and retinotopic mapping}

Each population of 2500 neural elements was arranged in a 50 by 50
grid, with positions on the grid representing a retinotopic
mapping similar to that found empirically both in the superior
colliculus \citep{ottes_visuomotor_1986} and in visual cortex
\citep{eric_l._schwartz_computational_1980} and assumed in
this work to persist throughout the oculomotor system.

In a retinotopic mapping, the Cartesian co-ordinates of the light-sensitive
cells in the retina, whose density varies with distance from the fovea,
are transformed into the Cartesian co-ordinates of
the correspondingly active cells on the colliculus. The mapping ensures
that an even density of cells can be maintained in the colliculus, but
ensures that a group of adjoining, active, retinal neurons will
always activate an adjoining group of neurons on the collicular
surface.

The mapping turns out to resemble polar co-ordinates. That is, one
axis of the collicular surface specifies the eccentricity of
a retinal location (how far it is from the fovea) and the second axis
specifies the rotational angle of the retinal location; we therefore
use the convention of referring to the eccentricity axis on the
colliculus as $r$ and the rotation axis as $\phi$.

The \e{cortical magnification factor}, $M(r)$, gives the relationship
between the
radial eccentricity $r$ and the retinal neural density. As in
\cite{cope_basal_2017}, we use a first-order approximation of the form
for $M(r)$ given in  \cite{rovamo_estimation_1979}:
\begin{equation} \label{eq:cmf}
M(r) = \frac{M_f}{1+\frac{r}{E_2}}
\end{equation}
The foveal magnification, $M_f$, is the magnification of the most central
region of the retina and has a value in the human of about
7.8 mm/\dg~\citep{rovamo_estimation_1979}.

In our model, $M_f$ is related to $W_{nfs}$, the width of the retinotopic neural
field, $W_{fov}$, the width of the eye's field of view and $E_2$, the
eccentricity at which the retinal density has halved by:
\begin{equation} \label{eq:fm}
   M_f = \frac {W_{nfs}} {E_2\;\ln\left(\frac{W_{fov}}{2 E_2} + 1\right)}
\end{equation}
Here, $W_{nfs}$ is 50 (the side length of the 50x50 grid)
and $W_{fov}$ is set to 61\dg, a
reduction from the biophysically accurate 150\dg~due to the small
number of neurons in the retinotopic neural field. $E_2$ is
2.5 \citep{cope_basal_2017,slotnick_electrophysiological_2001}.

The mapping from the retinotopic co-ordinates in the brain to rotational
co-ordinates of the stimulus/response was written down by
\cite{schwartz_spatial_1977,eric_l._schwartz_computational_1980}
for measurements of
striate cortex [visual stimulus to electrophysiological response---\cite{daniel_representation_1961, talbot_physiological_1941}]
and by \cite{ottes_visuomotor_1986} for superior colliculus data
[electrophysiological SC stimulus to eye movement
response---\cite{robinson_eye_1972}].
We used the following statement of this mapping to introduce stimuli
into the `World' input population of the brain model:
\begin{equation}\label{eq:thetas_to_phi}
   \phi = \frac{W_{nfs}} {2 \pi}\arctan\left(\frac{\theta_{y}^{t}}{\theta_{x}^{t}}\right)
\end{equation}
\begin{equation}\label{eq:thetas_to_r}
   r = M_fE_2\,\ln\left(\frac{1}{E_2}\sqrt{{\theta_{x}^{t}}^2 + {\theta_{y}^{t}}^2}+1\right)
\end{equation}
Note that we use $r$ and $\phi$ as the co-ordinates on the `collicular
surface'. Schwartz uses $r$ and $\phi$ as the polar coordinates of the
retinal stimulus; Ottes et al.~use $r$ and $\phi$ as polar coordinates
for the eye movement response; both use $u$ and $v$ as the Cartesian co-ordinates of the neural map. We use $\theta_{x}^{t}$ and $\theta_{y}^{t}$
to give Euler rotations for the retinal target stimulus. Note also that
the form of Eqns.~\ref{eq:thetas_to_phi} \& \ref{eq:thetas_to_r} is slightly
different from that given in
\cite{ottes_visuomotor_1986} because our $\theta_{x}^{t}$ and
$\theta_{y}^{t}$ are not the polar co-ordinates used in that work.

The mapping encompases the entire visual field; the value of
$\phi$ is allowed to vary from 0\dg~to 360\dg~along its axis.
Effectively, the two contralateral colliculi found in the biology are
incorporated into a
single, square map, avoiding the need to carry out the kind of `colliculus
gluing' described in \cite{tabareau_geometry_2007}.

It is straightforward to show that the reverse mapping is given by:
\begin{equation}
   \theta_x = E_2 \left(e^{\frac{r}{M_f E_2}} - 1\right).\cos\left(\frac{2 \pi \phi}{W_{nfs}}\right)
\end{equation}
\begin{equation}
   \theta_y = E_2 \left(e^{\frac{r}{M_f E_2}} - 1\right).\sin\left(\frac{2 \pi \phi}{W_{nfs}}\right)
\end{equation}
where we have dropped the $t$ superscript on $\theta_x$ \& $\theta_y$, as
these equations transform a collicular location into rotations of the eye.

Fig.~\ref{fig:mapping} shows the result of the mapping for a view of
two cross-shaped luminances. One cross illuminates the fovea, which
results in a large comb-shape of activity. The more peripheral cross
produces (in FEF) an indistinct object centred at a larger value of
$r$.

\subsubsection{Network}

Briefly, the model consists of input from the World population (see
Fig.~\ref{brain_model}, green population box) producing activity
in an `express' pathway to superior colliculus (purple) and
simultaneously in cortex, represented here by the FEF population (grey
boxes in Fig.~\ref{brain_model}). The express pathway causes short
latency activity in the superficial superior colliculus, which
directly innervates the deeper layers of the superior colliculus
(SC\_deep). Activity in FEF generates firing in a
thalamo-cortico-basal ganglia loop. The output of the basal ganglia is
the substantia nigra pars reticulata (SNr) which tonically inhibits
SC\_deep. If a location of activity in FEF is able to dominate
selection in the basal ganglia circuit, the corresponding location in
SNr will dis-inhibit and activity will build up in SC\_deep encoding
the saccade end point.

Connections shown in red are one to one connections; dark blue
projections indicate a connectivity pattern which `fans out' with a
2-D Gaussian kernel; lighter blue connections from the subthalamic
nucleus (STN) to SNr and globus pallidus externum (GPe)
are diffuse, all-to-all connections and projections coloured green are
one-to-one connections that decay towards the fovea so that foveal
activity in FEF does not swamp the basal ganglia which would prevent
peripheral luminances from ever being selected.
%
Note that SC\_deep contains two recurrent connections; one is
excitatory, with a Gaussian kernel mapping and the other implements
tecto-tectal inhibition, which increases the inhibition between
activity in opposite hemispheres of the field of view \citep{gian_g._mascetti_tectotectal_1981,olivier_evidence_2000} helping
to resolve competition between saccades to the left and right. The
tecto-tectal inhibitory connection is \e{not} present in
the \ccg~model. In all other respects the model is as described in
\cite{cope_basal_2017}.
%
We have not listed the parameters of the network in tabular form here,
instead, the reader is referred to the SpineML declarative
specification of the model from the link given in SUPPLEMENTAL DATA.
The easiest way to access this information is by using SpineCreator.


\subsection{Brainstem model}

We implemented a saccadic burst generator (SBG) based
on the connectivity outlined in \cite{gancarz_neural_1998}. The SBG network
for two of the model's
six channels is shown in Fig.~\ref{sbg}. Activity from the output
layer of superior colliculus (SC\_avg) is fed into each channel, which
sums the activity it receives and processes it in populations each of
a single neural element representing all the neurons in that
population. Each channel of the SBG functions to create the motor neuron
activations that are required to accelerate the eye in a particular
direction, then hold the eye in its new position against the returning
force generated by the elastic properties of the muscles.  The
required motor neuron activations are therefore a combination of features:
a brief burst of
increased activity that accelerates the eye; followed by a period
of activity that is less than the burst firing rate but higher than
the tonic rate that exists when the eye is at the centre. This
holds the eye in its new position.

The SBG connectivity produces each of the these features separately,
then sums them to create the desired `bump and tonic' activation
time series.  The input to the first population in the SBG, the
long-lead burst neurons (LLBNs), is conceived as originating from
one of the deep layers of the superior colliculus.  The activity
of the LLBNs are passed to excitatory burst neurons (EBNs) which,
in turn, inhibit the LLBNs via the activity of the inhibitory
burst neurons (IBNs).  This feedback loop has a transmission delay,
which allows activity to build up in the EBNs before the inhibition
is activated and the activity is then reduced again.  This mechanism
generates the `bump'.

The generation of the `tonic' phase of the required time series is
achieved simply by integrating the bump over time and multiplying by a
some small gain factor.  This is the function of the tonic neurons (TNs).
The firing rate of the motor neuron defines the amount of force applied
to the eye by that muscle.  Thus, the size of the `bump' defines how far
the eye moves in that channel's direction.  The gain and delay parameters
in the LLBN-EBN-IBN-LLBN feedback loop therefore have to be tuned such that
the endpoint of the saccade is reasonably accurate.  Furthermore the
restoring force generated by the elasticity of the muscles is dependent
on the radial distance.  The value of the new tonic firing rate, after
the `bump' is dependent on the end location of the eye.  If the ratio
between the EBN firing rate and the TN firing rate is not exactly
correct, the eye will drift away from the saccade endpoint after the
saccade has been completed.  The EBN-TN connection strength is
therefore tuned such that the TN firing rate yields a stable eye
position across a range of eye eccentricities.

The omnipause neurons (OPNs) are tonically active and inhibit the EBNs.
The activity of the OPNs is itself inhibited by activity in the LLBNs.
The purpose of this arrangement is to ensure the eye does not move in
response to neural noise.

Each mean activity of all the neurons in each SBG population (except the
TNs) is defined by a single leaky integrator, first order differential equation.
\begin{equation}\label{eq:LIN}
   \frac{da}{dt} = \frac{1}{\tau}(y-a)
\end{equation}
where $a$ is the activation of the nucleus, and $\tau$ is the time constant
of the nucleus. $y$ is a piecewise linear function of the weighted sum of
inputs to the nucleus and is given by
\begin{equation}
       y(IN) = \begin{cases}
       0    & IN \leq b \\
      IN-b   & b \leq IN \leq 1+b \\
      1   & IN \geq 1+b \\
   \end{cases}
\end{equation}
where $b$ is the $IN$ axis offset.  $IN$ is the weighted sum of inputs to
the nucleus and is given by,
\begin{equation}
    IN = \sum_{m}^{M} w_{mn} a_{m}
\end{equation}
where $a_{m}$ is the activation of the $m^{th}$ afferent nucleus.
$w_{mn}$ is the connection strength between the $m^{th}$ afferent nucleus
and the current nucleus. The activity of the TNs are defined as
\begin{equation}
   \frac{da}{dt} = \frac{1}{\tau}\,y
\end{equation}
with an identical piecewise linear transfer function as the other SBG populations.

\subsection{biomechanical eye}

The output signals of the brainstem are used to drive the biomechanical model.
The latter is not only used to get tangible feedback on the simulated saccades
including motion trajectories, but adds one more modelling dimension related to
the inertial properties of the eye plant including muscle properties.

The biomechanical eye model, implemented using the OpenSim framework
\citep{seth_opensim:_2011}, is anatomically represented by a sphere of
uniform mass distribution. The diameter of the eye is 24 mm for adults,
with small variations between individuals; the mass of the eye is 7.5 grams.
The eyeball is actuated by six extraocular muscles
(EOMs). The EOMs are arranged in three pairs forming a cone inside the
orbit with the apex being located inside the cranium in a tendonous
ring called the annulus of Zinn. An important feature of the
oculomotor system which greatly affects its overall behavior is the
existence of dynamic EOM pulleys. Their role is to guide the pivot
point of the EOMs. In our model, a pulley for each EOM has been
modeled by a point on the orbit whose location depends on the current
eye orientation.

The force applied by EOMs is controlled by an excitatory signal
supplied by motoneurons in the brainstem. The neural drive to produce a saccadic
eye movement can be characterized by a pulse component to overcome the
viscoelasticity of the orbital plant, a step component to stabilize the eye in
the new position, and a slide component that models the gradual transition
between the pulse and step.

The dynamics of muscular
forces can be split into: 1) The elasticity of the muscles. 2) A delay
between the onset of the afferent excitatory signal and the actual
muscle contraction, caused by the transmission time of the action
potentials and by the necessary calcium release at the muscle
fibres. We developed a custom extraocular muscle model which captures
these features.

Passive forces due to the fatty tissues inside the eye orbit also
affect eye dynamics. Their role is critical in eliminating the
influence of head and body movements. We incorporated a custom torque, $\mathbf{t}$,
which acts like a rotational spring-damper apparatus, resisting
eyeball movements. It has elastic and viscous properties governed by
$\mathbf{t} = -K\mathbf{R}-C\mathbf{U}$ where $\mathbf{R}$ is the
eye's orientation and $\mathbf{U}$ is its angular velocity. $K$ and
$C$ are constants. A fuller description of the biomechanical model
can be found in \cite{papapavlou_physics-based_2014}.

\subsection{Integrating the models and closing the loop}

The \ccg~model closed its loop by passing the centroid of activity in
SC\_deep (once it had surpassed a threshold) back to the code that
controlled the world, which would then use this location to
instantaneously change the model's view of the world. In our extended
model, it was necessary to connect the output of the brain model back
to its input via the saccadic burst generator model and the
biomechanical eye. The resulting state of the eye, rather than the
centroid of the superior colliculus, was used to compute the input to
the brain, given the luminances visible in the world.

A number of studies have considered the form of the connection between
the deeper layers of the superior colliculus and the saccadic burst
generator \citep{van_gisbergen_experimental_1985,ottes_visuomotor_1986,waitzman_superior_1991,groh_converting_2001,arai_two-dimensional_1994,goossens_dynamic_2006,tabareau_geometry_2007,van_opstal_linear_2008,goossens_optimal_2012},
which has become known as the spatial temporal transform (STT).  The
spatial aspect of the transform is thought to be implemented by a
weight-mapping \citep{tabareau_geometry_2007,arai_two-dimensional_1994} and we
follow this idea.
Arai and co-workers trained a 20x20 neural network model of the
superior colliculus to discover the weight map under the assumption of
2D Gaussian activation profiles~\citep{arai_two-dimensional_1994}.
The training approach of \cite{arai_two-dimensional_1994} was not feasible in this study
due to the length of time required to run our model and its
stochasticity, which meant multiple runs of the model were necessary
in order to generate output statistics.
\cite{tabareau_geometry_2007} wrote down a theoretical form of the weight
map, which follows from the mapping of \cite{ottes_visuomotor_1986}
and the assumption of invariant 2D Gaussian activity profiles in
SC. As they found it closely resembles the
results of \cite{arai_two-dimensional_1994}, and it is a simple
formulation, we considered
it as the means to generate the six weight maps in our own model.
One barrier to the use of the weight map
in \cite{tabareau_geometry_2007} was the \ccg~model's violation of the
\e{invariant integral hypothesis}. This states that the number of spikes
emitted by a neural element during a saccade (or in our model, the
integral of the neuron's output during the saccade) should be a
function only of its position within the hill of collicular
activity. That is, for any time-dependent hill of activity
$\mathcal{A}(\mathbf{z},t)$ at $\mathbf{z} = (r,\phi)$ on the collicular
surface, the integrated activity  $A_{\mathbf{x}}$ in an element at a
vector $\mathbf{x}$ away from $\mathbf{z}$ is
\begin{equation}
A_{\mathbf{x}} = \int_t \mathcal{A}(\mathbf{z}-\mathbf{x}, t)\;dt
\end{equation}
which is invariant for all $\mathbf{z}$. However, the very mapping on
which the \cite{tabareau_geometry_2007} result is based leads to a
very \e{variant} activity profile in the \ccg~model. A luminance of
a given size which excites activity near to the fovea causes activity
in a large number of neurons, whereas activity far from the fovea
excites a much smaller region. This effect is clearly demonstrated in
Fig.~\ref{fig:mapping} for equal sized targets both on and distal from
the fovea.

This led us to hypothesize that the retinotopic mapping be accompanied
by an associated widening projection field such that the hill of
activity in superior colliculus is invariant with position on the
collicular surface. There are a number of locations in the system in
which this widening projection field could exist. It could be
implemented in the projections between the retinal populations and the
superficial layer of SC along with the projection between the World
and the FEF population. However, this would affect activity within the
basal ganglia of the model, contradicting a result
in \cite{cope_basal_2017} which explains the `hockey stick' profile
for saccade latency as a function of saccade eccentricity. Instead, we
suggest that a widening projection field is encoded within the
superior colliculus itself, a complex, multi-layered structure which
could quite plausibly support such a function. Indeed, such widening
activity can be seen in the stimulation experiments
in \cite{vokoun_intralaminar_2010}
and \cite{vokoun_response_2014}. Although in this work we do not model
the SC in detail, we extended the model with a third functional layer named SC\_deep2, shown in Fig.~\ref{scdeep}
(\ccg~has only the two layers SC\_sup and SC\_deep). We
introduced a widening projection based on a Gaussian projection field
whose width, $\sigma(r)$ varies in inverse proportion to the
magnification factor, $M(r)$, given in Eq.~\ref{eq:cmf} according to:
\begin{equation} \label{eq:sigmar}
\sigma(r) = \frac{m_{\sigma}}{M(r)} - \frac{m_{\sigma}}{M^0} + \sigma_0 \qquad r > r_0
\end{equation}
$m_{\sigma}$ is a scalar parameter which determines the `magnitude
of the widening'. $M^0$ is the `starting' magnification factor;
within the foveal region ($0 \leq r \leq r_0$), the projection field
is not allowed to widen and so
\begin{equation} \label{eq:sigmar2}
\sigma(r) = \sigma_0 \qquad r \leq r_0
\end{equation}
which makes $\sigma_0$ the width of the Gaussian projection field
within the foveal region. (Note that the value chosen for the width
of the foveal region, $r_0$ is not identical to the foveal shift parameter used
in the \e{DecayingAtFovea} projections into striatum.)
The \e{Widening Gaussian} projection weight, $w(r,d)$ is then
computed as:
\begin{equation} \label{eq:widening}
w(r,d) = e^{-\frac{d^2}{2\sigma\left(r\right)^2}}
\end{equation}
where $d$ is the distance between the source and destination
elements in the collicular plane.  $m_\sigma$ was set to 50, $\sigma_0$
was 0.3, $M^0$ was 12.43 and $r_0$ was 20.

A further issue regarding the use of the theoretical weight map
in \cite{tabareau_geometry_2007} was that it does not consider the
existence of the oblique extraocular muscles. There is evidence that
only two dimensional information is encoded in superior
colliculus \citep{wurtz_activity_1972,hepp_monkey_1993,van_opstal_two-rather_1991}, but the
eye is actuated by six extraocular muscles. In order to find out a
possible form for the input to the oblique muscles we carried out a
training process
which depended on a
centroid computation in SC\_deep. For the four rectus muscles, the
resulting weight maps resembled those found by
\cite{arai_two-dimensional_1994}. The trained
maps for the oblique muscles had a form very close to those for the
inferior and superior rectus channels, but with a smaller magnitude.
The inferior oblique map resembled the superior rectus map and the
superior oblique map resembled the inferior rectus. When parameterising
the theoretical weight maps, we set the inferior/superior
oblique maps to be 1/10$^{\mathrm{th}}$ of the superior/inferior rectus
maps, respectively. Interestingly, this suggests that there is a built-in
synergy between the vertical and oblique channels in the eye, although
the results will show there is some systematic change in the oblique error
with saccade end-point location.

\cite{tabareau_geometry_2007} gives a formulation for the weight maps
in which it is possible to project both a positive and a negative
weight. In our model, all projections from SC\_deep are excitatory.
This means that each channel has a weight which follows the form:
\begin{equation} \label{eq:weightmaps}
w(r,\phi) = i\;e\,^{jr}\;\sin\left(\frac{2\pi\phi}{W_{nfs}} + k\right)
\end{equation}
where $i$, $j$ and $k$ are per-channel parameters for the weight
maps. $k$ is determined by the mapping. Only the positive part of the
sine is utilised. $i$ and $j$ are parameters to be found.

The saccadic burst generator model was originally conceived with the
assumption of a step input, which returns to zero activity at a suitable
time to curtail the saccade and avoid staircase
saccades \citep{gancarz_neural_1998}. In our model
there is no such mechanism to reduce activity in SC\_deep, and elsewhere.
Although a successful, accurate saccade towards a target luminance will
remove the excitation
which caused the activity in SC\_deep by bringing the target luminance
within the masked,
foveal region, the activity in SC decays too slowly to avoid
additional saccadic movements. We found it necessary to hypothesize
an inhibitory feedback mechanism from the SBG to the brain model.
This is shown in Fig.~\ref{sbg}, which indicates how the output from
the inhibitory burst neurons (IBN) of the SBG model are used to feed
back an inhibitory signal to the SC\_deep, thalamus and FEF populations
in the brain model, resetting them ready for the next saccade.

The output signals from the six channels of the SBG were connected to
the six motoneuron inputs of the biomechanical eye. The signal was
normalised; a value of 1 meaning that all the motoneurons in the
output population were firing at their maximum rate and the force
exerted by the relevant extraocular muscle was maximal. Channels
innervated extraocular muscles as follows: Up: superior rectus; Down:
inferior rectus; Right: medial rectus; Left: lateral rectus; Z+:
superior oblique; Z-: inferior oblique. Because the medial rectus
induces a rightward rotation of the eye, our single virtual eye is
a \e{left} eye. The OpenSim implementation of the biomechanical eye
was `wrapped' (in the software sense) in a BRAHMS component. This made
it possible to integrate the OpenSim model into the BRAHMS
framework. The wrapper ensured that the input and output signals were
correctly transferred and, importantly, handled the disparity in the
solver timesteps used in the OpenSim model (25~ms) and the neural
model (1~ms). This was achieved by having the BRAHMS wrapper create a
separate thread to run the OpenSim model. The BRAHMS wrapper component
was called on each 1 ms timestep, receiving the instantaneous
activations from the motoneurons in the SBG. These activations, and
the current simulation time, were written into a shared memory area,
accessible by the OpenSim thread. Running independently, the OpenSim
thread would update its inputs (using the most recent values in the
shared memory area) whenever the simulation time had increased by 25
ms. It would then recompute its outputs (the rotational state of the
eye) and write these into the same shared memory. The BRAHMS wrapper
would update its outputs whenever they were changed in the shared
memory by the OpenSim thread. A direct connection of the six outputs
of the BRAHMS eye model component to the six inputs of the
worldDataMaker BRAHMS component was specified in
the \stob~external.xsl file.

The eye model outputs its rotational state at each
timestep. The rotational state is used to compute the view of the
world in the eye's frame of reference. To simplify
the calculation, the luminances exist on a spherical surface at the
centre of which is the eye. A hand-coded BRAHMS component called
worldDataMaker computes the projection of the luminances into the
eye's frame of reference and then converts this representation into a
retinotopic map to pass into the brain model. The input to the brain
model is thus able to change continuously, on every timestep, rather
than in a step-wise fashion when a saccade occurs, as in the \ccg~model.

In the worldDataMaker BRAHMS component, the rotational state of the
eye was used to construct Euler rotation matrices which transformed
between the world's frame of reference and the eye's frame of
reference. The worldDataMaker component received a specification of
the world luminances in a JSON file called luminances.json at the
start of each simulation. luminances.json specified the position,
shape, size, luminance, appearance time and disappearance time of an
arbitrary number of luminances. With this information, the
instantaneous rotational state of the eye and the parameters of the
retinotopic transform, it was able to compute the instantaneous input
to the brain model.

The final models, on which the results of this paper are based are
named `TModel3', `TModel4' and `TModel5'. Descriptions of these, and earlier
versions of the model can be
found in the code repository given in SUPPLEMENTAL DATA.


\section{Results}

\subsection{Weight maps}

We found the best parameters for the exponential in Eq.~\ref{eq:weightmaps}
($i$ and $j$) by a manual tuning process. After selecting values for $i$ and $j$
in either the horizontal or vertical/oblique channels, we ran the model 6 times
at each of 8 target eccentricities (7\dg--14\dg) which were purely
in the direction of the
newly parameterised channel. The training saccades were produced as described below in
Sect.~\ref{sec:singlesaccades}, with the same fixation and target luminances
(crosses of magnitude 0.2 and 0.3) but with
the fixation offset and target onset occurring at 0.2~s. We measured the end-point
of the saccade by
detecting the location at which the saccade velocity had dropped below
0.005 of its peak. We iterated until the mean saccade endpoint plotted versus target
was close to the ideal straight line---see Fig.~\ref{sacc_vs_targ}(a) \& (c). We
applied the same parameters to both directions of each channel;
$i_{up} = i_{down} = 0.00195$, $j_{up} = j_{down} = 0.075$,
$i_{left} = i_{right} = 0.0016$ and $j_{left} = j_{right} = 0.067$. The resulting
weight maps (where the oblique maps are 1/10$^{\mathrm{th}}$ of the vertical maps,
as described earlier) are shown in Fig.~\ref{weightmaps}.

\subsection{Saccade accuracy}

In Fig.~\ref{sacc_vs_targ}, we showed
the result of running the model to targets located on the principle
axes, on which the model was trained. We then simulated single saccades
to targets in one hemifield of the eye's field of view, with
eccentricities between 6\dg~and 14.5\dg. As in the
training, we ran the simulation 6 times for each target,
$\mathbf{\theta}^t = (\theta_{x}^t, \theta_{y}^t, 0)$ to obtain
mean saccade end-points.
Fig.~\ref{errorsurfaceTM3} shows saccade accuracy results for an
entire hemifield in the na\"ive model which passed the output of
SC\_deep directly to SBG via the weight maps. The ratio of
the magnitude of the error vector to the magnitude of the target
vector is plotted using a colour map. This ratio is shown for the full,
three dimensional error vector in Fig.~\ref{errorsurfaceTM3}(a) and
for the $x$, $y$ and $z$ components in Figs.~\ref{errorsurfaceTM3}(b)--(c).
Inspection of  Fig.~\ref{errorsurfaceTM3}(a) shows that the end-point
error is minimal along the principle axes ($\theta_{x}^t=\;$0 or
$\theta_{y}^t=\;$0) and maximal near the 45\dg~oblique targets
(blue lines) with the end point error as high as 80\% of the
programmed saccade magnitude.
The $x$ component error map in Fig.~\ref{errorsurfaceTM3}(b)
shows the same trend, mirrored about the `Target X' axis, whereas
the $y$ and $z$ component errors are, relatively, much smaller.
Because the $x$ component of the error is clearly contributing to
end point errors which would not be considered `on target',
especially for oblique saccades, we considered the effect of the
non-uniform size of the hill of activity in SC\_deep.

In our model, the location, \e{size} and shape of activity in FEF,
the basal ganglia, thalamus and superior colliculus is eccentricity
dependent, in line with the retinotopic mapping stated by
\cite{ottes_visuomotor_1986}. More eccentric targets generate reduced
activity, because fewer retinal neurons are excited far from the
fovea. \cite{cope_basal_2017} showed that this relationship can explain
increased saccadic latencies for distal targets, resulting from
reduced activity in the decision making circuitry of the basal ganglia.
However, the notion that activity in superior colliculus is
eccentricity-dependent conflicts with the result of
\cite{tabareau_geometry_2007}, who showed
that an invariant hill of activity was required if this complex
logarithmic weight mapping was to be used to drive a
two-degree-of-freedom saccadic burst generator, and also with
experimental findings, which do not show significant eccentricity
dependence, at least in the burst layer~\citep{anderson_two-dimensional_1998}.

To bring our model in line with these results, whilst maintaining the
eccentricity dependent activity in basal ganglia, we hypothesised
that a `widening projection' exists between two maps in superior
colliculus. Activities in one SC\_deep layer remains eccentricity-dependent,
with loops back to thalamus and cortex and through basal ganglia.
This activity is then fed through a projection, which applies a
Gaussian projection field, whose width increases with increasing
stimulus eccentricity according to Eq.~\ref{eq:widening}.
The activity in this second SC\_deep layer is then fed to the
weight maps of the SBG. This model was called `TModel4'. TModel4
was parameterised such that its horizontal and vertical error
was similar---so that its equivalent of Fig.~\ref{sacc_vs_targ}
showed a similar sum of squares error.

Figs.~\ref{errorsurface}(a)--(d) show the same percentage errors
for TModel4 as Fig.~\ref{errorsurfaceTM3} shows for TModel3.
First of all, note that the error magnitudes are much smaller. The mean
errors are smaller for every axis. The largest errors produced by
the model are approximately 15\%, which are
within the boundaries of what some authors have suggested would be regarded
as an accurate saccade \citep{mcpeek_saccade_2002,mcpeek_incomplete_2006}.
The magnitude of the largest error vector is approximately 1.5\dg.

This result indicates that the
exponential part of the Ottes et al.~weight map from SC to the SBG cannot
on its own compensate for the eccentricity-dependent size of the hill of
activity. The introduction of a widening projection field substantially
improves the mean accuracy of saccades across the field of view.
We therefore suggest that the transformation between retinotopically mapped
activity, and eccentricity-independent activity width occurs
within the superior colliculus and works alongside a simple,
monotonically increasing weight map between SC and the SBG channels.

\subsection{Single saccades} \label{sec:singlesaccades}

Having finalised the model by setting the weight maps, we then proceeded
to exercise the model (TModel4), starting with saccades to a single target;
prosaccades. Fig.~\ref{outmany}(a) shows 9 representative saccades
to a single
target luminance. Initially, the eye had rotational state
$\theta_x=\theta_y=\theta_z=0$ with its fovea directed at a
fixation luminance cross (span 6\dg, bar width 2\dg) of
magnitude 0.2 (in arbitrary units). At a
simulation time of 0.4~s, the fixation luminance was set to 0 and
a target luminance cross of the same dimensions as the fixation but with
magnitude 0.3 was illuminated at one of the
9 different locations, marked by crosses in Fig.~\ref{outmany}(a).
The resulting trajectories are plotted, with colour indicating the
relationship between trajectories and target crosses. The approximate
end-point error is visible in this figure, although the last point in
each trajectory is the saccade position at 0.8~s and not the
velocity-based end-point described above. Figs.~\ref{outmany}(b) and (c)
show the rotational components of the blue and red trajectories
in Fig.~\ref{outmany}(a) along with the target and fixation
luminance values. Rotations are the eye's Euler rotational components
in the world frame of reference.

\subsection{Saccade Latencies}

To verify that our implementation of the brain model has the same
functionality as that reported in \cite{cope_basal_2017}, we
investigated the effect on saccadic response times of:
target eccentricity; and any gap or overlap between fixation off-time
and target on-time. We showed that the full model reproduces the
`hockey stick' shape shown in Fig. 7 of \cite{cope_basal_2017} for
horizontal [Fig.~\ref{lat_vs_all}(a)], vertical
[Fig.~\ref{lat_vs_all}(b)] and oblique saccades (not shown). The latency
increases with eccentricity far from the fovea because the retinotopic mapping
reduces the activity in the basal ganglia for more eccentric targets (this
effect is described in detail in \cite{cope_basal_2017}). Closer to the fovea,
the interaction between the foveal mask and the activity in FEF again leads
to reduced input into into the basal ganglia and an increased time to
achieve disinhibition in SNr.

Fig.~\ref{lat_vs_all}(c) shows latencies achieved when varying the time
between fixation offset and target onset. This is termed the \e{gap condition};
and is represented by a scalar value which, if positive, refers to a gap
between fixation offset and target onset, and when negative, signifies an
overlap, with the fixation luminance persisting past the time at which the
target is illuminated. A negative gap is also termed an \e{overlap}.
Again, we verify the behaviour
presented in \cite{cope_basal_2017}, explained as resulting from the
inhibition of the cortico-thalamic loop by SNr. In the gap condition, when
the fixation luminance is removed, activity in STN immediately begins to
decay, allowing SNr activity to reduce and thereby reducing inhibition on
thalamus, allowing the target luminance to build up quickly in FEF, thalamus
and through the basal ganglia's striatum and SNr. The shape of the curves in
Fig.~\ref{lat_vs_all}(c) matches the results in \cite{cope_basal_2017}
for target luminances of 1 and 0.6; for overlaps longer than 100~ms
(gap $<$ -100~ms), the latency becomes constant; the saccade is programmed
whilst the fixation is present, with the target luminance inducing sufficient
activity in striatum to `break through' the SNr inhibition caused by the
fixation. If the target luminance is reduced to 0.3, the balance is altered
in favour of the fixation and the latency vs.~gap becomes approximately
linear and equal to the overlap time plus around 100~ms.

Fig.~\ref{lat_vs_all}(d) shows the effect of the dopamine parameter on
saccade latencies in gap, step and overlap conditions. In general, the effect of
decreasing the dopamine parameter was a smooth, monotonic and undramatic
increase in saccade latency. However, the data for the overlap condition
with a target luminance which was 3 times as bright as the fixation luminance
was more interesting. Here we see a transition around a dopamine value if 0.7.
Below this value,
the basal ganglia is not able to select the target luminance until the fixation
is removed, reducing the excitatory drive from STN to SNr, and consequently
the inhibition from SNr to the thalamo-cortical loop. For the target
luminance 0.6, 0.7 dopamine allows the basal ganglia to select sufficiently well
so that the target can build up in the thalamo-cortical loop, in spite of
the fixation overlap.

The relationship between latency and the target luminance is given in
Fig.~\ref{lat_vs_all}(e). This shows latency for a 100~ms gap, step and
100~ms overlap conditions for a given fixation luminance of 0.2,
and a horizontally located target at $\theta_{y}^{t}=$-10\dg. For the
gap condition, we see very short latencies for luminances of about 0.75 and above.
Finally, the activity driving these express saccades is initiated by high firing rates
in the superficial layer of SC (SCs), which then drives activity in thalamus and
through the basal ganglia. A gradual transition from express saccades to
reflexive saccades is observed as the contribution of the SCs becomes weaker
and the drive from FEF into the thalamo-cortical loop becomes necessary
to elicit a saccade.
%
A similar gradual transition, albeit for higher latencies
is seen for the step condition. At higher target luminances, the SCs has a
greater effect on the activity in the thalamo-cortical loop. However, the
activity in STN caused by the fixation luminance increases the latency at
all luminance values compared with the gap condition.
%
The overlap condition leads to increased latencies for luminances below 2.5, but
meets the step condition above this value, at which the 0.2 fixation
luminance appears to have a negligible effect on the system.

\subsection{Saccade sequences}

We now present results derived from the fully parameterised and integrated
model; where we took advantage of the fact that it is a closed loop system. This
allowed us to present sequences of target luminances and allow the model to
direct its fovea at the most salient target.

\subsubsection{Out \& return}

We investigated the behaviour of the model for saccade sequences.
In one experiment, we illuminated a fixation
cross from 0~s until 0.4~s, followed by a target at (0,-10\dg)
from 0.4~s until 0.8~s. Finally, the fixation was again shown from 0.8~s
until the end of the simulation at 2~s. This induced a saccade to a
10\dg~eccentricity, followed by a return saccade back to the
null point. We noticed some irregularities in the return saccades, which
though surprisingly accurate, had a significant overshoot.
More perplexingly, if the target was switched repeatedly between 0\dg~and
10\dg, second and subsequent \e{outward} saccades also showed this
overshoot. We found that the cause of these irregularties was the
lack (in `TModel4') of any mechanism to reset the tonic neurons in the
SBG after the first saccade. This resulted in TN activity in the left
channel \e{and also} in the right channel. Interestingly, this ensured
that, at  least for a few, consecutive out-and-return saccades, the saccade accuracy was
relatively good, with trajectories resembling experimental data
(\cite{bahill_trajectories_1979}, p.~6). % Poss. figure.
Nevertheless, the lack of a reset of TN activity was an oversight, and
is indeed proposed and included in the connectivity of the
\cite{gancarz_neural_1998} model. We implemented this feature by adding an
additional inhibitory input to the `integrator' component of TModel4,
driven by the contralateral EBN population, naming the new model `TModel5'.
Now, when the eye is directed towards an eccentric target which is then
exchanged with a target at the null point, the EBN activity toward the
null point will tend to extinguish the TN activity which was holding
the eye at the eccentric position.
We verified that none of the single saccade results were affected
by this modification.

Fig.~\ref{outrtn} shows the
outward and return trajectories produced by the experiment
with the TN reset mechanism. Panel (a)
shows the $x$ and $y$ rotation trajectory; panel (b) shows individual
rotational components of the eye. Fig.~\ref{outrtn}(c) shows
out and return trajectories for three other saccade targets; horizontal,
vertical and oblique. The trajectories have characteristic shapes and
also show some stochastic variation caused by the noise in the model [see
dashed trajectories in Fig.~\ref{outrtn}(a)].

The return trajectories (magenta lines) showed a
distinctly different form from the outward trajectories. They overshot
their destination (the null point) significantly. This resulted from the
removal of the TN activity which was holding the eye at the eccentric
target location. Removal of this activity, and thus the static force
exerted by the corresponding extraocular muscle, meant that the eye
was subject both to a new muscular force towards the null point
\e{alongside} the restorative spring force of the lengthened rectus
muscle. This stands as a shortcoming of the model.

\subsubsection{Double steps}

In another experiment, we probed the response of the model to double step
stimuli of the type described in \cite{becker_analysis_1979}. In that
work, the response of human subjects was investigated when shown stimuli
at 15\dg~and 30\dg~eccentricity with variable delay between
the stimuli. If the smaller eccentricity stimulus was shown first, followed
by the more distal on the same side of the field of view, this was called
a `staircase' presentation.
We carried
out a `staircase' presentation, shown in Fig.~\ref{doublesteps},
where our small eccentricity luminance was at 8\dg~and our more
distal luminance was at 12\dg~(both to the right of centre). We
found that there was a critical time delay between the luminances
of about 30~ms. If they were presented with a delay smaller than this
value, then a single, slightly hypermetric saccade was made. This
response type is called a \e{final angle response}. A delay greater
than 30~ms between the stimuli would lead to double step saccades
(a so-called \e{initial angle response}),
with the first saccade arriving at 8\dg~(though with greater
variability than normal), and a second saccade being made
to a location hypometric of 12\dg~after a pause of about 240~ms.
Fig.~\ref{doublesteps}(a) shows the mean trajectories from 5 simulations
of the staircase doublestep presentation alongside the result for a
single saccade to the final angle of 12\dg. Dash-dot lines show
$\pm$1 standard deviation about the mean. The corresponding trajectories
are shown in Fig.~\ref{doublesteps}(b).

Inspection of the activity maps in FEF and SC\_deep (not shown) indicates
that when
the 8\dg~target is illuminated for 30~ms or more, the activity
associated with this target angle is able to dominate the activity,
hence the execution of a reasonably accurate saccade. The inhibitory
feedback from the SBG then extinguishes activity in FEF, thalamus and
SC, which means that a full 200~ms or more is required to allow
activity in these populations to build up again in order to make the
smaller saccade from 8\dg~to 12\dg. This is in contrast
to experimental findings in which the corrective second saccade is
often executed \e{more quickly} than if it were programmed on its
own~\citep{becker_analysis_1979}.


\section{Discussion}

The aim of this study was to demonstrate the importance of modelling
neurological systems \e{in concert with} the biomechanical systems
with which they have evolved in parallel. We hypothesised that by combining
existing neurophysiological models with an accurate
model of a musculo-skeletal system, and then `closing the loop' by allowing
the movements of the virtual muscles to modulate sensory feedback to
the brain model, shortcomings in the constituent models would be
revealed, leading to new knowledge. To demonstrate the validity
of this hypothesis, we built an integrated model and then identified the
modifications which were necessary to give it the ability to make
accurate movements under one type of stimulus. We then
examined its behaviour with other stimuli.

We chose the oculomotor model as a basis for this study because it has
only three degrees of freedom, making it one of the simplest
musculo-skeletal systems. Furthermore, eye movements fall into several
well-defined categories, each being controlled by separate
brain circuits, we were therefore justified in modelling a
system which produced only saccadic eye movements. Nevertheless, we are
aware that we did not create a complete model of the system; no treatment
of the cerebellum was attempted, justified because cerebellum appears to have
only a minor effect on saccade accuracy \citep{dean_adaptive_2008}, probably correcting
for slow to medium timescale changes in the physical dynamics of the
eyeball \citep{dean_learning_1994}.

To summarise our model integration: We combined
the \ccg~model \citep{cope_basal_2017} with a saccadic burst generator
model based on the work of \cite{gancarz_neural_1998}, using this to
drive the input of a new biomechanical eye model. To achieve the spatial
transformation from the retinotopic maps of the \ccg~model to the six
`muscle channel' inputs for the saccadic burst generator, we used
the mapping of \cite{ottes_visuomotor_1986} to produce parameterised
weight maps along with an empirically discovered synergy for the torsional
weight maps. We introduced an additional
transformation to the brain model to achieve invariant sized hills of
activity in superior colliculus to fulfil the invariant integral
hypothesis of \cite{tabareau_geometry_2007}. We closed the loop using
a software component which transformed a view of a world containing
luminous cross shapes into the eye's frame of reference, given its
instantaneous rotational state. This component also computed the inverse
of the mapping from \cite{ottes_visuomotor_1986} to project the view
retinotopically into the brain model.
%
This paper serves to describe how we achieved the integration in
order to test our hypothesis, and we intend that the material and
methods section, along with the model code itself, will help others
to carry out similar studies. However, we wish to devote the majority
of this discussion to what can be learned from an integrated model
of a combined brain and biomechanical system, using our oculomotor system
as an example.

Our integration approach revealed three ways in which this model fails
to provide a full understanding of the saccadic system. In each case,
the issue is made clear \e{as a result of the integration}. This is not
to say that other approaches may not also reveal shortcomings; we will
see that one of our cases has been independently identified
\citep{groh_effects_2011}.

\subsection{The need for a widening projection field}

The original combination of the \ccg~model with the theoretical weight
maps of \cite{ottes_visuomotor_1986} and \cite{tabareau_geometry_2007}
resulted in a model which was
able to produce accurate saccades only along the principle rotational
axes (Fig.~\ref{errorsurfaceTM3}). Thus, \e{the integration of
the models} suggested that an additional
layer was required to achieve accurate saccades for oblique, as well
as for horizontal and vertical saccades. Although the \e{need} for an
invariant integral is discussed in \cite{tabareau_geometry_2007} as resulting from
their theoretical study, the mechanism by which such an invariant
Gaussian hill is generated is not. By combining the models, we were
forced to consider this mechanism, and hypothesised that a widening
projection field would be a candidate mechanism. The results of
Fig.~\ref{errorsurface} indicate that a substantial improvement
in accuracy is indeed achieved by this new mechanism.

\subsection{Saccades from non-null starting positions}

The implementation of a biophysically accurate model of the eye, and
the closed-loop nature of the model makes it very natural to consider
how the model will behave making saccades from arbitrary starting
positions, or how it would respond to a sequence of stimuli. This was
the motivation for the out-and-return experiment (Fig.~\ref{outrtn})
as well as for the double step experiment (Fig.~\ref{doublesteps}).
We found that return saccades were substantially affected by the
biomechanics of the eye, as the brain and brainstem model had no
mechanism to account for the position-dependent restoring forces applied
by the eye. This question has been addressed by other authors;
\cite{groh_effects_2011} investigates the effect of
initial eye position on stimulated saccades
and finds a need for the signal in superior colliculus to be modulated
by an eye position signal. \cite{ling_effects_2007} shows the existence
of a position dependent firing rate offset in abducens neurons. Though
we will not speculate here on the mechanism by which return saccades
may be made accurate whilst also resetting the activity of tonic
neurons in the SBG, it is interesting that in the model in which we
omitted to reset TN activity (TModel4), we obtained relatively accurate
out-and-return saccades which closely resembled experimental data.
We suggest that residual activity in TN populations may offer an
explanation for how the restorative force exerted by
the elastic oculomotor muscles is compensated for. A comparison of
this idea with that of \cite{groh_effects_2011} (that there is a
modulation, from a brainstem signal, of the SC readout) would make
a subject for a future study. Although these existing studies have
highlighted this issue, the inaccurate return saccades which the
model makes from eccentric starting positions provide a clear example
of the way in which integrating known models into a closed-loop system
can highlight deficiencies in the model.

\subsection{Inhibitory feedback from saccadic burst generator to brain}

The third issue raised by the integration of the component models of
the saccadic system has, like the return saccades, to do with resetting
activity. In this case, rather than the reset of activity in the TN
population in the brainstem, it is the question of how the activity
in the \e{brain} model should be reset after each saccade. When a target
luminance is projected onto the World population in the model, this
induces activity which `reverberates' in loops through FEF, basal
ganglia, SC and thalamus. The brainstem contains a mechanism to limit
the timescale of a saccade (inhibitory feedback from EBN, via IBN to
LLBN; see Fig.~\ref{sbg}). However, if the activity in SC is not
reset, then following the completion of the first saccade, a series
of subsequent `staircase' saccades will be executed. There needs to
be a mechanism to extinguish activity in SC, but also in FEF and
thalamus, as activity in either of these populations can build up
and eventually cause repeat activity in SC and another saccade. We
added hypothetical inhibitory feedback connections to our model,
such that the IBN populations in the SBG would inhibit activity in
FEF, thalamus and SC\_deep (Fig.~\ref{sbg}), preventing the
occurrence of staircase saccades.

An examination of the behaviour of the model when presented with
`double-step stimuli' reveals a problem with this scheme. We found
that when double-step stimuli were presented (where an initial
target at 8\dg~was replaced with a 12\dg~target after 30 or 40~ms)
and a double saccade was made [Fig.~\ref{doublesteps}(a), black
lines] the second saccade latency was \e{longer} even than the
initial saccade. This contrasts with \cite{becker_analysis_1979} who
find that second, corrective saccades occur with \e{shorter} latencies.
This suggests that the inhibitory reset signal implemented in
this model is too strong or has the wrong timescales. This issue
highlights the fact that connections \e{between} component models
are quite as important as the connections within each model.

\subsection{Concluding remarks}

% Cerebellum omitted
The omission of the cerebellum will not have escaped the reader's notice.
Whilst many of the nuclei known to be involved in the production of
saccadic eye movements are incorporated within the model, the cerebellum
is not. The cerebellum is known to play an important r\^ole in saccade
programming
\citep{dean_learning_1994, schweighofer_model_1996,quaia_extent_2000,kleine_saccade-related_2003}.
It may be able to completely replace the functionality of the colliculus
when lesioned \citep{aizawa_reversible_1998,lefevre_distributed_1998}.
However, this r\^ole is typically considered to be one of accuracy tuning
\citep{barash_saccadic_1999,dean_learning_1994}; operating as an additive
model. Furthermore, saccades made by individuals with cerebellar ataxias
perform with only moderate loss of saccade accuracy
\citep{barash_saccadic_1999,federighi_differences_2011}. Because we
did not address learning in our model, and because our aim was to
demonstrate the utility of integrating brain with biomechanics in order
to highlight deficiencies, we considered the omission of the cerebellar
nuclei acceptable in the present work.

% Component stretching/main sequence not addressed
We have not addressed the question of saccade duration in this paper.
Saccade duration is of interest in models which produce two (or three)
dimensional saccades, because the dynamics of a saccade follow well known
relationships with the saccade eccentricity, regardless of the saccade
angle. This causes a problem for models (such as the present one) for which
some of the dynamic behaviour is generated within orthogonal components.
For example, saccade duration increases with target eccentricity. A
10\dg~eccentricity oblique (45\dg~up and right) saccade is composed
(approximately) of a 7\dg~upwards component and a 7\dg~rightwards
component. If the component based model is responsible for the dynamics,
then the 10\dg~oblique saccade would be expected to have the dynamics
of a 7\dg~up or 7\dg~right saccade. This is not found in practice, and
the components are said to have been stretched, hence the name for this
effect `component stretching'. The \cite{gancarz_neural_1998} model is
reported to take account of the component stretching effect via the OPN
neuron population. We did not find this effect in our implementation of
the model; the duration of oblique saccades at a given eccentricity was
always substantially different from the duration of the corresponding
purely vertical or horizontal saccade. Because there is a somewhat
complicated interplay between the dynamics of the superior colliculus
driving the dynamic system of the SBG, we feel this is outside the scope
of the current work and a subject for a future paper.

% Significance of this work.
This work represents a step forward in the modelling of neuromuscular systems,
not because it significantly advances any of the constituent models, but
because it \e{integrates} the models into a complete, \e{behaving} system.
This is not the first integrated brain model composed of separately developed
components. The works of \cite{nguyen_saccade_2014} and
\cite{thurat_biomimetic_2015} are both based
on an example of a brain model which drives a simple, second order model of the
eye. \cite{dewolf_spiking_2016} describes a reach model which integrates models
of cortex and cerebellum to drive a two degree-of-freedom arm model. Both of
these example systems nevertheless operate using `curated' inputs supplied
by the modeller.

In contrast, the current work allows the state of the system to determine the
input delivered to the model. The modeller only curates the state of the world
at each time point, but the actual input to the model depends on the eye's
rotational state.
This is, to our knowledge, the first model which integrates the brain with an
accurate biophysical system and closes the loop in this way, enabling the
system to reproduce behaviour.
As such, it offers a platform for testing more complex saccadic behaviour such as
antisaccades or saccades in the presence of distractor stimuli. We believe
that by building closed loop systems which express behaviour, we, and others
will develop a new field of \e{computational neurobehaviour}, which will share
themes from neuroscience, artificial intelligence, decision science and embodied
robotics.
