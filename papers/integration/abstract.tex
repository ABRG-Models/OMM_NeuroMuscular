% While writing, don't stop for errors
\nonstopmode

%%% This is based on the Frontiers template.
\documentclass{frontiersSCNS}

\usepackage{url,hyperref,lineno,microtype}
\usepackage[english]{babel}
\usepackage{amsmath}
\usepackage[onehalfspacing]{setspace}
\linenumbers

% Trick to define a language alias and permit language = {en} in the .bib file.
% From: http://tex.stackexchange.com/questions/199254/babel-define-language-synonym
\usepackage{letltxmacro}
\LetLtxMacro{\ORIGselectlanguage}{\selectlanguage}
\makeatletter
\DeclareRobustCommand{\selectlanguage}[1]{%
  \@ifundefined{alias@\string#1}
    {\ORIGselectlanguage{#1}}
    {\begingroup\edef\x{\endgroup
       \noexpand\ORIGselectlanguage{\@nameuse{alias@#1}}}\x}%
}
\newcommand{\definelanguagealias}[2]{%
  \@namedef{alias@#1}{#2}%
}
\makeatother
\definelanguagealias{en}{english}
\definelanguagealias{eng}{english}
% End language alias trick


\def\keyFont{\fontsize{8}{11}\helveticabold }
\def\firstAuthorLast{James {et~al.}}
\def\Authors{Sebastian James\,$^{1,2,*}$, Alexander Blenkinsop\,$^{1,2}$, \\
  Alexander Cope\,$^3$, Sean Anderson\,$^{2,4}$, Dimitrios Stanev\,$^5$, \\
  Chris Papapavlou\,$^5$, Konstantinos Moustakas\,$^5$, Kevin Gurney\,$^{1,2}$}
% Affiliations should be keyed to the author's name with superscript
% numbers and be listed as follows: Laboratory, Institute, Department,
% Organization, City, State abbreviation (USA, Canada, Australia), and
% Country (without detailed address information such as city zip codes
% or street names).
%
% If one of the authors has a change of address, list the new address
% below the correspondence details using a superscript symbol and use
% the same symbol to indicate the author in the author list.
\def\Address{\\
$^{1}$Adaptive Behaviour Research Group, Department of Psychology,
  The University of Sheffield, Sheffield, UK \\
$^{2}$Insigneo Institute for in-silico Medicine,
  The University of Sheffield, Sheffield, UK \\
$^{3}$Department of Computer Science,
  The University of Sheffield, Sheffield, UK \\
$^{4}$Department of Automatic Control Systems Engineering,
  The University of Sheffield, Sheffield, UK \\
$^{5}$Department of Computer Science,
  The University of Patras, Patras, Greece \\
}
% The Corresponding Author should be marked with an asterisk. Provide
% the exact contact address (this time including street name and city
% zip code) and email of the corresponding author
\def\corrAuthor{Kevin Gurney}
\def\corrAddress{Department of Psychology, The University of Sheffield,
  Western Bank, Sheffield, S10 2TP, UK}
\def\corrEmail{k.gurney@sheffield.ac.uk}


\begin{document}

% A place for aliases:
% Aliases
\newcommand{\ccg}{Cope-Chambers-Prescott-Gurney}
\newcommand{\stob}{SpineML\_2\_BRAHMS}
% Emphasis and bold.
\newcommand{\e}{\emph}
\newcommand{\bol}{\textbf}

\newcommand{\blue}{\textcolor{blue}}

% I like my citations in blue
\newcommand{\mycite}[1]{\blue{\cite{#1}}}

% expon function is roman e
\newcommand{\expon}{\mathrm{e}}

% A font to indicate a branch of a github repository
\newcommand{\branch}[1]{\textbf{\texttt{#1}}}

\newcommand{\cmnt}[1]{\blue{#1}}

\newcommand{\dg}{$\degree$}


\onecolumn
\firstpage{1}

\title[Integrated brain and biomechanics]{
  Integrating brain and biomechanical models - a new  paradigm
  for understanding  neuro-muscular control
}

\author[\firstAuthorLast ]{\Authors}
\address{}
\correspondance{}
\extraAuth{} % If more than 1 corr. author refer to original template
             % and fill this in

\maketitle
%% END PREAMBLE %%%%%%%%%%%%%%%%%%%%%%%%%%%%%%%%%%%%%%%%%%%%%%%%%%%%%

\begin{abstract}
\section{}
To date, realistic models of how the central nervous system governs
behaviour have been restricted in scope to the brain, brainstem or
spinal column, as if these existed as disembodied organs.
Further, the model is often exercised in relation to an in vivo
physiological experiment with input comprising an impulse, a periodic
signal or constant activation, and output as a pattern of neural
activity in one or more neural populations. Any link to behaviour is
inferred only indirectly via these activity patterns.
%
We argue that to discover the principles of operation of neural
systems, it is necessary to express their behaviour in terms of
physical movements of a realistic motor system, and to supply inputs
that mimic sensory experience. To do this with confidence, we must
connect our brain models to neuro-muscular models and provide relevant
visual and proprioceptive feedback signals, thereby closing the loop
of the simulation.
%
This paper describes an effort to develop just such an integrated
brain and biomechanical system using a number of pre-existing
models. It describes a model of the saccadic oculomotor system
incorporating a neuromuscular model of the eye and its six extraocular
muscles. The position of the eye determines how illumination of a
retinotopic input population projects information about the location
of a saccade target into the system.
%
A pre-existing saccadic burst generator model was incorporated into
the system, which generated motoneuron activity patterns suitable for
driving the biomechanical eye. The model was demonstrated to make
accurate saccades to a target luminance under a set of environmental
constraints.
%
Challenges encountered in the development of this model showed the
importance of this integrated modelling approach. Thus, we exposed
shortcomings in individual model components which were only apparent
when these were supplied with the more plausible inputs available in a
closed loop design. Consequently we were able to suggest missing
functionality which the system would require to reproduce more
realistic behaviour.
%
The construction of such closed-loop animal models constitutes a new
paradigm of {\em computational neurobehaviour} and promises a more
thoroughgoing approach to our understanding of the brain's function as
a controller for movement and behaviour.

\tiny
 \keyFont{
   \section{Keywords:}
   integrated brain biomechanics neuromuscular oculomotor saccade basal ganglia
 } % 5 to 8 keywords
\end{abstract}

\end{document}
